\documentclass[letterpaper]{article}

\usepackage{amsmath}
\usepackage{url}
\usepackage{mathtools}
\newcommand\defeq{\stackrel{\mathclap{\normalfont\mbox{\tiny def}}}{=}}

\begin{document}

\section{Is Equation 9 correct?}

Recall Equation 9 in Heller and Ghahramani
(2005)\footnote{\url{http://www2.stat.duke.edu/~kheller/bhcnew.pdf}} is

$$
p(\mathbf{x} \mid \mathcal{D}) = \sum_{k \in \mathcal{N}} \omega_k p(\mathbf{x} \mid \mathcal{D}_k)
$$

where $\mathcal{N}$ is the set of all nodes in the tree, $\omega_k \defeq r_k \prod_{i \in
\mathcal{N}_k} (1 - r_i)$ is the weight on cluster $k$, and $\mathcal{N}_k$ is the set of the nodes
on the path from the root node to the parent node $k$.

We're beginning to suspect this is wrong, and that $\omega_k$ is not correctly defined. To sample a
point from the posterior predictive distribution, since BHC is a mixture model where each node is a
component, there must be some
process of sampling a node of BHC before sampling from that node. This would
result in the weighted sum in Equation 9 above and is implied by both $\omega_k$ and one of Heller's
presentations.\footnote{Last slide of
\url{http://www.gatsby.ucl.ac.uk/~heller/present/bhc/Ranbhc.ppt}}

Specifically, the paper describes this process as ``recursing through the tree starting at the root
node.'' So if $r_k$ is our probability of stopping at node $k$ during this recursion, $\omega_k$
represents the probability of ending up at node/component $k$ if we do \emph{not} stop at any of
the nodes along the path from the root of the tree to node $k$'s parent ($1 - r_i$), and we
\emph{do} stop at node $k$ ($r_k$).

\textbf{However}, $\omega_k$ doesn't seem to include any notion of choosing between left and right
children.  It's calculated based on probabilities assuming a fixed path from the root to the node,
$\mathcal{N}_k$. Thus, we think $\omega_k$ is an overestimate of the true value, and the
implementation confirms this: the sum of $\omega_k$ as defined in Equation 9 is consistently $> 1$
for BHC trees, while $\omega_k$ \emph{must} sum to 1 (see Section~\ref{sec:omegak}).

To be accurate, we think $\omega_k$ needs to include the probability of recursing from the root
node to node $k$ and \emph{not} descending into the other subtrees along the path.

\section{Proposed fix}

Stuart Sale's pyBHC
package\footnote{\url{https://github.com/stuartsale/pyBHC/blob/master/pyBHC/bhc.py#L181-L187}}
doesn't include functionality for evaluating the likelihood of a new point according to the
posterior predictive distribution, but \emph{does} have functionality for sampling from the
posterior predictive (Heller's BHC
code\footnote{\url{http://www.gatsby.ucl.ac.uk/~heller/code/bhc/}}
doesn't). In his procedure, he starts at the root node and chooses between the left and right
children with probability proportional to the ratio of the sizes of the left and right subtrees.
Specifically, if $n_k$ is the number of leaves under node $k$, at node $k$ he descends into the
left child with probability $n_{\text{left}} / (n_k)$ and the right
child with 1 minus that probability. So in this case,

$$
\omega_k \defeq r_k \prod_{i \in \mathcal{N}_k} (1 - r_i) R_{\mathcal{N}_k}(i)
$$

where the existing symbols are defined as before, and

\begin{align*}
  R_{\mathcal{N}_k}(i) = \begin{cases}
  n_{\text{Left}(i)} / n_i  & \text{if $\text{Left}(i) \in \mathcal{N}_k$ or $\text{Left}(i) = k$} \\
  n_{\text{Right}(i)} / n_i & \text{otherwise}
  \end{cases},
\end{align*}


i.e.\ $R_{\mathcal{N}_k}(i)$ is the ratio of the sizes of the children of a node in
$\mathcal{N}_k$, where the ratio is dependent on which child is also in the path $\mathcal{N}_k$.

To show this is true, we would ideally like to do the following three things:

\begin{enumerate}
  \item verify that the posterior predictive distribution with $\omega_k$ computed as above makes sensible
    predictions,
  \item derive this equation by ``rearranging the sum over all tree-consistent
    partitionings into a sum over all clusters in the tree'', and
  \item verify with one of the people who originally worked with BHC (e.g.\ Yang Xu or Katherine
    Heller) that our intuitions are correct and that Equation 9 is wrong.
\end{enumerate}

\section{$\omega_k$ must sum to 1}
\label{sec:omegak}

If $p(x)$ is a valid probability distribution, then $\int p(x) \; dx = 1$. Therefore,

\begin{align*}
1 &= \int p(\mathbf{x} \mid \mathcal{D}) \; d\mathbf{x} \\
  &= \int \sum_{k \in \mathcal{N}} \omega_k p(\mathbf{x} \mid \mathcal{D}_k) \; d\mathbf{x} \\
  &= \sum_{k \in \mathcal{N}} \omega_k \int p(\mathbf{x} \mid \mathcal{D}_k) \; d\mathbf{x} \\
  &= \sum_{k \in \mathcal{N}} \omega_k. \\
\end{align*}

So in BHC (and \emph{any} mixture model), the weights must sum to 1.

% The Dirichlet-Multinomial model used by BHC in our case simplifies to a Dirichlet-Categorical,
% which simplifies to a product of Beta-Bernoulli models. TODO: write-up exactly why this is.

\end{document}
